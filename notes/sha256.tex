\documentclass{report}

%%%%%%%%%Packages Start%%%%%%%%%
%%Basic packages
\usepackage[utf8]{inputenc}
\usepackage[english]{babel}
\usepackage{enumerate}
\usepackage[margin=1in]{geometry}
\usepackage{hyperref}
\usepackage{datetime}
%\usepackage{datetime2}

%%Code Packages
%\usepackage{color}
%\usepackage{xcolor}
%\usepackage{listings}
%\usepackage{minted}
%\usepackage{attachfile}
%\usepackage{accsupp}
%\usepackage{verbatim}
%\usepackage[misc]{ifsym}

% math
\usepackage{amsfonts}
\usepackage{amssymb}
\usepackage{amsmath}
\usepackage{amsthm}

% graphics
\usepackage{float}
%\usepackage{tikz}
%\usepackage{graphicx}
%\usepackage{pgfplot}
%\usepackage{fancyhdr}
%%%%%%%%%Packages End%%%%%%%%%%%

\author{Mimanshu Maheshwari}
\title{\textbf{SHA Hashing Notes}}
\date{\today, \currenttime} 


\begin{document}
\maketitle
\tableofcontents
\listoftables

\chapter{SHA 256}
\section{Introduction}
SHA256 is a 256 bits hash. Ment to provide 128 bits of security against collision attack. 
\section{Implementation}
SHA256 operates in a manner of MD4, MD5 and SHA-1. 
The message to be hashed is 
\begin{enumerate}

	\item{Padded with its length in such a way that the result is multiple of 512 bits long.}
	\item{Parsed into 512 bits message blocks $M^{1}, M^{1}, \ldots, M^{1},$}
	\item{Message blocks are processed one block at a time: Beginning with a fixed initial hash value	$H^{(0)}$, sequentially compute}
		\[ H^{(i)} = H^{(i-1)} + C_{M^{(i)}}(H^{(i-1)}) \]
		where $C$ is the SHA-256 \textit{compression function} and $+$ means word-wise $\mod 2^{32}$ addition. $H^{(N)}$ is the \textit{\textbf{hash}} of $M$.
\end{enumerate}

SHA-256 operates on 512-bits \textit{message block} and a 256-bits \textit{intermidiate hash value}. 
It essentially is a 256-bit cypher algorithm which encripts intermidiate hash value using the message block as key. 
Hence, their are two main components: 
\begin{itemize}
	\item{Compression Function}
	\item{message schedule}
\end{itemize}

\begin{center}
	\begin{table}[h!]
		\centering
		\begin{tabular}{|| c | c ||}
			\hline 
			\textbf{Notation} & \textbf{Meaning} \\ 
			\hline 
			$\oplus$ & Bitwise XOR \\
			\hline 
			$\vee$ & Bitwise AND \\ 
			\hline 
			$\wedge$ & Bitwise OR \\
			\hline 
			$\neg$ & Bitwise Complement\\ 
			\hline 
			$+$ & $\mod 2^{32}$ addition \\
			\hline 
			$R^{n}$ & right shift by \textit{n} bits\\
			\hline 
			$S^{n}$ & right rotate by \textit{n} bits\\
			\hline
		\end{tabular}
		\caption{Notation Reference}
		\label{notation-reference}
	\end{table}
\end{center}

All of the operators in \ref{notation-reference} table act on 32-bit words.

The initial value of $H^{(0)}$ is the following sequence of 32 bit words (which are obtained by taking the fractional parts of the square roots of the first eight primes.)

\begin{align}
	H_{1}^{(0)} &= 6a09e667 \\
	H_{2}^{(0)} &= bb67ae85 \\
	H_{3}^{(0)} &= 3c6ef372 \\
	H_{4}^{(0)} &= a54ff53a \\
	H_{5}^{(0)} &= 510e527f \\
	H_{6}^{(0)} &= 9b05688c \\
	H_{7}^{(0)} &= 1f83d9ab \\
	H_{8}^{(0)} &= 5be0cd19
\end{align}

\section{Preprocessing}
Computing the hash of message begins by padding the message: 
\begin{enumerate}
	\item{Pad the message in usual way:}
		Suppose the lenght of message $M$, in bits, is $l$. Append the bit $"1"$ to the end of message, and the the $k$ zero bits, where $k$ is the smallest non-negative solution to the equation $l + 1 + 1 \equiv 448 \mod 512$. To this append the 64-bit block which is equal to the number $l$ written in binary. For example, the (8-bit ASCII) message "abc" has length $8 \cdot 3 = 24$ so it is padded with a one, then $448 - (24 + 1) = 423$ zero bits, and thenthe length to become the 512-bit padded message: 
		\[ 01100001\ \ 01100010\ \ 01100011\ \ \underbrace{0000\ldots0}_{423-bits}\ \ \overbrace{00\ldots011000}^{64-bits} \]
		The length of the padded message should now be 512 bits.
\end{enumerate}

\appendix
\end{document}
% https://tex.stackexchange.com/a/44838
% https://www.overleaf.com/learn/latex/Mathematical_expressions
% https://www.overleaf.com/learn/latex/Tables
% https://mirror.niser.ac.in/ctan/obsolete/info/math/voss/mathmode/Mathmode.pdf
