\documentclass{report}

%%%%%%%%%Packages Start%%%%%%%%%
%%Basic packages
\usepackage[utf8]{inputenc}
\usepackage[english]{babel}
\usepackage{enumerate}
\usepackage[margin=1in]{geometry}
\usepackage{hyperref}
\usepackage{datetime}
%\usepackage{datetime2}

%%Code Packages
%\usepackage{color}
%\usepackage{xcolor}
%\usepackage{listings}
%\usepackage{minted}
%\usepackage{attachfile}
%\usepackage{accsupp}
%\usepackage{verbatim}
%\usepackage[misc]{ifsym}

% math
\usepackage{amsfonts}
\usepackage{amssymb}
\usepackage{amsmath}
\usepackage{amsthm}

% graphics
\usepackage{float}
%\usepackage{tikz}
%\usepackage{graphicx}
%\usepackage{pgfplot}
%\usepackage{fancyhdr}
%%%%%%%%%Packages End%%%%%%%%%%%

\author{Mimanshu Maheshwari}
\title{\textbf{SHA Hashing Notes}}
\date{\today, \currenttime} 


\begin{document}
\maketitle
\tableofcontents
\listoftables

\chapter{SHA 256}
\section{Introduction}
SHA256 is a 256 bits hash. Ment to provide 128 bits of security against collision attack. 
\section{Implementation}
SHA256 operates in a manner of MD4, MD5 and SHA-1. 
The message to be hashed is 
\begin{enumerate}

	\item{Padded with its length in such a way that the result is multiple of 512 bits long.}
	\item{Parsed into 512 bits message blocks $M^{1}, M^{1}, \ldots, M^{1},$}
	\item{Message blocks are processed one block at a time: Beginning with a fixed initial hash value	$H^{(0)}$, sequentially compute}
		\[ H^{(i)} = H^{(i-1)} + C_{M^{(i)}}(H^{(i-1)}) \]
		where $C$ is the SHA-256 \textit{compression function} and $+$ means word-wise $\mod 2^{32}$ addition. $H^{(N)}$ is the \textit{\textbf{hash}} of $M$.
\end{enumerate}

SHA-256 operates on 512-bits \textit{message block} and a 256-bits \textit{intermidiate hash value}. 
It essentially is a 256-bit cypher algorithm which encripts intermidiate hash value using the message block as key. 
Hence, their are two main components: 
\begin{itemize}
	\item{Compression Function}
	\item{message schedule}
\end{itemize}

\begin{center}
	\begin{table}[h!]
		\centering
		\begin{tabular}{|| c | c ||}
			\hline 
			\textbf{Notation} & \textbf{Meaning} \\ 
			\hline 
			$\oplus$ & Bitwise XOR \\
			\hline 
			$\vee$ & Bitwise AND \\ 
			\hline 
			$\wedge$ & Bitwise OR \\
			\hline 
			$\neg$ & Bitwise Complement\\ 
			\hline 
			$+$ & $\mod 2^{32}$ addition \\
			\hline 
			$R^{n}$ & right shift by \textit{n} bits\\
			\hline 
			$S^{n}$ & right rotate by \textit{n} bits\\
			\hline
		\end{tabular}
		\caption{Notation Reference}
		\label{notation-reference}
	\end{table}
\end{center}

All of the operators in \ref{notation-reference} table act on 32-bit words.

The initial value of $H^{(0)}$ is the following sequence of 32 bit words (which are obtained by taking the fractional parts of the square roots of the first eight primes.)

\begin{align}
	H_{1}^{(0)} &= 6a09e667 \\
	H_{2}^{(0)} &= bb67ae85 \\
	H_{3}^{(0)} &= 3c6ef372 \\
	H_{4}^{(0)} &= a54ff53a \\
	H_{5}^{(0)} &= 510e527f \\
	H_{6}^{(0)} &= 9b05688c \\
	H_{7}^{(0)} &= 1f83d9ab \\
	H_{8}^{(0)} &= 5be0cd19
\end{align}

\section{Preprocessing}
Computing the hash of message begins by padding the message: 
\begin{enumerate}
	\item{Pad the message in usual way:}
		Suppose the lenght of message $M$, in bits, is $l$. Append the bit $"1"$ to the end of message, and the the $k$ zero bits, where $k$ is the smallest non-negative solution to the equation $l + 1 + 1 \equiv 448 \mod 512$. To this append the 64-bit block which is equal to the number $l$ written in binary. For example, the (8-bit ASCII) message "abc" has length $8 \cdot 3 = 24$ so it is padded with a one, then $448 - (24 + 1) = 423$ zero bits, and thenthe length to become the 512-bit padded message: 
		\[ 01100001\ \ 01100010\ \ 01100011\ \ \underbrace{0000\ldots0}_{423-bits}\ \ \overbrace{00\ldots011000}^{64-bits} \]
		The length of the padded message should now be 512 bits.
	\item{Parse the message into $N$ 512-bits block $M^{(1)}, M^{(1)}, \ldots , M^{(1)}$}
		The first 32 bits of message block $i$ are denoted $M^{(i)}_{0}$, the next 32 bits are $M^{(i)}_{1}$, and so on up to $M^{(i)}_{15}$. We use big-endian convention througout, so within each 32-bit word, the left most bit is stored in the most significant bit position.
\end{enumerate}

\section{Main loop}
The hash computation proceeds as follows: \\
$for\ i = 1 \rightarrow N$ ( $N =$ Number of blocks in the padded message) 

\begin{itemize}
	\item{Initialize registers\ $ a, b, c, d, e, f, g, h $}
		with the $(i- 1)^{st}$ intermidiate hash value ($=$ initial hash value when $i = 1$)

		\begin{align*}
			a &\leftarrow H_{1}^{(i - 1)} \\
			b &\leftarrow H_{2}^{(i - 1)} \\
			c &\leftarrow H_{3}^{(i - 1)} \\
			d &\leftarrow H_{4}^{(i - 1)} \\
				&\vdots \\
			h &\leftarrow H_{8}^{(i - 1)}
		\end{align*}

	\item{Apply the SHA-256 \textit{compression function} to update registers $a, b, c, \ldots, h$} \\
		$for\ j = 0 \rightarrow to 63$ \\
		Compute $Ch(e,f,g), Maj(a,b,c), \sum_{0}{(a)}, \sum_{1}{(e)}, and W_{j}$
		\begin{align*}
			T_1 &\leftarrow h + \sum_{1}{(e)} + Ch(e, f, g) + K_{j} + W_{j} \\
			T_2 &\leftarrow \sum_{0}{(a)} + Maj(a, b, c) \\
			h &\leftarrow g \\ 
			g &\leftarrow f \\ 
			f &\leftarrow e \\ 
			e &\leftarrow d + T_{1} \\ 
			d &\leftarrow c \\ 
			c &\leftarrow b \\ 
			b &\leftarrow a \\ 
			a &\leftarrow T_{1} + T_{2}
		\end{align*}
	\item{Compute the $i^{th}$ intermidiate hash value $H^{(i)}$}
		\begin{align*}
			H_{1}^{(i)} &\leftarrow H_{1}^{(i - 1)} \\
			H_{2}^{(i)} &\leftarrow H_{2}^{(i - 1)} \\
			H_{3}^{(i)} &\leftarrow H_{3}^{(i - 1)} \\
			H_{4}^{(i)} &\leftarrow H_{4}^{(i - 1)} \\
									&\vdots \\
			H_{8}^{(i)} &\leftarrow H_{8}^{(i - 1)}
		\end{align*}
\end{itemize}

$ H^{(N)} = \left( H_{1}^{(N)}, H_{2}^{(N)}, H_{3}^{(N)}, \ldots, H_{8}^{(N)} \right)  $ is the hash of $M$.
\section{Definations}
Six logical functions are used in SHA-256. Each function operates on 32-bits words and produces a 32-bit word as output.

\begin{align}
	Ch(x,y,z)       &= (x \wedge y) \oplus (\neg x \wedge z)\\
	Maj(x,y,z)      &= (x \wedge y) \oplus (x \wedge z) \oplus (y \wedge z) \\
	\sum_{0}{(x)}   &= S^{2}(x)     \oplus S^{13}(x)    \oplus S^{22}(x) \\
	\sum_{1}{(x)}   &= S^{6}(x)     \oplus S^{11}(x)    \oplus S^{25}(x) \\
	\sigma_{0}{(x)} &= S^{7}(x)     \oplus S^{18}(x)    \oplus R^{3}(x) \\
	\sigma_{1}{(x)} &= S^{17}(x)    \oplus S^{19}(x)    \oplus R^{10}(x)
\end{align}

\subsection{Expanded Message Blocks}
$W_0, W_1, \ldots, W_{63}$ computed as follows via the \textbf{SHA-256 message schedule}: \\
$ W_{j} = M_{j}^{(i)}$ for $j = 0, 1, 2, \ldots, 15,$ and \\
$for\ j = 16 \rightarrow 63$ 
\begin{align*}
	W_{j} \leftarrow \sigma_{1}(W_{(j - 2)}) + W_{(j - 7)} + \sigma_{0}(W_{(j - 15)}) + W_{(j - 16)}
\end{align*}

A sequence of contant words $K_{0}, K_{1}, K_{2}, \ldots, K_{63}$ is used in SHA-256. in Hex, these are given by: 
\begin{table}[h!]
	\centering
	\begin{tabular}{|| c  c  c  c ||}
		\hline\hline
		0x428a2f98 & 0x71374491 & 0xb5c0fbcf & 0xe9b5dba5 \\
		0x3956c25b & 0x59f111f1 & 0x923f82a4 & 0xab1c5ed5 \\
		0xd807aa98 & 0x12835b01 & 0x243185be & 0x550c7dc3 \\
		0x72be5d74 & 0x80deb1fe & 0x9bdc06a7 & 0xc19bf174 \\
		0xe49b69c1 & 0xefbe4786 & 0x0fc19dc6 & 0x240ca1cc \\
		0x2de92c6f & 0x4a7484aa & 0x5cb0a9dc & 0x76f988da \\
		0x983e5152 & 0xa831c66d & 0xb00327c8 & 0xbf597fc7 \\
		0xc6e00bf3 & 0xd5a79147 & 0x06ca6351 & 0x14292967 \\
		0x27b70a85 & 0x2e1b2138 & 0x4d2c6dfc & 0x53380d13 \\
		0x650a7354 & 0x766a0abb & 0x81c2c92e & 0x92722c85 \\
		0xa2bfe8a1 & 0xa81a664b & 0xc24b8b70 & 0xc76c51a3 \\
		0xd192e819 & 0xd6990624 & 0xf40e3585 & 0x106aa070 \\
		0x19a4c116 & 0x1e376c08 & 0x2748774c & 0x34b0bcb5 \\
		0x391c0cb3 & 0x4ed8aa4a & 0x5b9cca4f & 0x682e6ff3 \\
		0x748f82ee & 0x78a5636f & 0x84c87814 & 0x8cc70208 \\
		0x90befffa & 0xa4506ceb & 0xbef9a3f7 & 0xc67178f2 \\
		\hline\hline
	\end{tabular}
	\caption{First 32 bits of the fractional part of the cube roots of the first 64 primes}
	\label{contant-words}
\end{table}

use \ref{contant-words} table to set initial value of buffer.
	
\appendix
\end{document}
% https://tex.stackexchange.com/a/44838
% https://www.overleaf.com/learn/latex/Mathematical_expressions
% https://www.overleaf.com/learn/latex/Tables
% https://mirror.niser.ac.in/ctan/obsolete/info/math/voss/mathmode/Mathmode.pdf

